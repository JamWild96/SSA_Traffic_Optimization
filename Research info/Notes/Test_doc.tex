\documentclass{article}
\usepackage{amsmath}
\usepackage{graphicx}
\usepackage{algorithm}
\usepackage{algpseudocode}
\usepackage{listings}
\usepackage{xcolor}

\lstset{language=C++,
basicstyle=\ttfamily\small,
keywordstyle=\color{blue},
commentstyle=\color{green!50!black},
stringstyle=\color{red},
showstringspaces=false,
frame=single}

\title{CUDA-Based Implementation of Sparrow Search Algorithm (SSA) for Traffic Congestion Control}
\author{Jam Wilder \\
Department of Computer Science \\
Central Washington University \\
\texttt{jamarius.wilder@cwu.edu}}

\begin{document}

\maketitle

\section{Introduction}
Traffic congestion control is crucial for urban management. The Sparrow Search Algorithm (SSA), a metaheuristic inspired by the social behavior of sparrows, efficiently addresses optimization issues related to congestion management by balancing exploration and exploitation.

\section{Sparrow Search Algorithm (SSA)}
The SSA consists of two types of sparrows: producers and scroungers. Producers explore the search space, while scroungers exploit existing good solutions. The mathematical representation of sparrow updates is:
\begin{equation}
X_i^{t+1} = X_i^t + Q \times (X_{\text{best}}^t - X_i^t) + R
\end{equation}
where $X_i^t$ is the current solution, $X_{\text{best}}^t$ the best solution at iteration $t$, and $Q$, $R$ are random factors.

\section{Optimization Problem Formulation}
For traffic congestion, the objective function may be defined as:
\begin{equation}
\text{Minimize } Z = \alpha T + \beta Q + \gamma E
\end{equation}
with parameters representing travel time ($T$), queue length ($Q$), emissions ($E$), and weighting factors $\alpha$, $\beta$, $\gamma$.

\section{CUDA Implementation Overview}
CUDA parallelization significantly accelerates SSA performance by simultaneously evaluating multiple sparrow positions.

\subsection{CUDA Parallelization Approach}
Each CUDA thread evaluates one sparrow solution. Blocks manage groups of sparrows, and the GPU handles parallel computation of fitness functions.

\begin{algorithm}
\caption{CUDA Parallel SSA Implementation}
\begin{algorithmic}[1]
\State Initialize sparrow positions (traffic parameters)
\State Copy data from host (CPU) to device (GPU)
\While {termination criterion not satisfied}
\State Launch kernel: evaluate fitness of solutions
\State Update global best solution using parallel reduction
\State Launch kernel: update sparrow positions
\State Synchronize threads and copy best solution to CPU
\EndWhile
\State Return optimized traffic control parameters
\end{algorithmic}
\end{algorithm}

\subsection{CUDA Kernel Example}
Simplified CUDA kernel to evaluate fitness:

\begin{lstlisting}
global void evaluateFitness(float *solutions, float *fitness, int n) {
int idx = threadIdx.x + blockDim.x * blockIdx.x;
if (idx < n) {
// Evaluate traffic simulation model
fitness[idx] = simulateTraffic(solutions[idx]);
}
}
\end{lstlisting}

\section{Practical Advantages}

CUDA-based SSA implementation:
\begin{itemize}
\item Enables real-time traffic management
\item Reduces average waiting times and emissions
\item Enhances responsiveness to dynamic traffic conditions
\end{itemize}

\section{Challenges and Solutions}

\begin{tabular}{|p{5cm}|p{5cm}|}
\hline
\textbf{Challenges} & \textbf{Solutions} \ \hline
High dimensionality & Dimensionality reduction \ \hline
Real-time constraints & Parallel GPU computation \ \hline
Dynamic traffic changes & Adaptive SSA parameters \ \hline
\end{tabular}

\section{Conclusion}
Integrating SSA with CUDA for traffic management significantly improves computational efficiency and solution quality, making it suitable for real-time traffic congestion control.

\end{document}
